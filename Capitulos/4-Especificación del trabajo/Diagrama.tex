% ==== Diagrama de arquitectura del modelo híbrido SNN-CNN ====
% Para ser incluido desde el documento principal con % ==== Diagrama de arquitectura del modelo híbrido SNN-CNN ====
% Para ser incluido desde el documento principal con % ==== Diagrama de arquitectura del modelo híbrido SNN-CNN ====
% Para ser incluido desde el documento principal con % ==== Diagrama de arquitectura del modelo híbrido SNN-CNN ====
% Para ser incluido desde el documento principal con \input{Capitulos/4-Especificación del trabajo/Diagrama}

\begin{figure}[h]
\centering
\begin{tikzpicture}[scale=0.8]

% Definiciones de colores específicas para el diagrama (manteniendo compatibilidad)
\definecolor{EntradaColor}{rgb}{1,0.6,0.2}
\definecolor{SNNColor}{rgb}{0.2,0.6,1}
\definecolor{ConvColor}{rgb}{0.8,0.3,0.8}
\definecolor{FusionColor}{rgb}{0.4,0.8,0.4}

% Estilo de conexión personalizado (corregido para evitar conflictos)
\tikzset{
    conexion/.style={ultra thick, ->, >=stealth, opacity=0.8}
}

% === Caja A - Entrada (Codificación de cuantiles) ===
\node[draw=black,
      fill=EntradaColor,
      minimum height=3cm,
      minimum width=1.5cm,
      rounded corners=3pt,
      text width=1.3cm,
      align=center] (A) at (0,0) {
        \small\textbf{Entrada}\\[0.2cm]
        \footnotesize Codificación\\
        cuantiles\\[0.1cm]
        \tiny N = len(cuantiles)-1\\
        \tiny $\sim$64 neuronas
};

% === Caja B - Capa SNN (LIF + STDP) ===
\node[draw=black,
      fill=SNNColor,
      minimum height=3.5cm,
      minimum width=2cm,
      rounded corners=3pt,
      text width=1.8cm,
      align=center] (B) at (4,0) {
        \small\textbf{Capa B}\\[0.2cm]
        \footnotesize SNN Base\\
        \footnotesize LIF + Recurrente\\[0.1cm]
        \tiny N = 100 neuronas\\
        \tiny threshold $\in$ [-65,-50]\\
        \tiny decay $\in$ [80,150]\\
        \tiny STDP: nu1, nu2 $\in$ [-0.5,0.5]
};

% === Caja C - Capa Convolucional Híbrida ===
\node[draw=black,
      fill=ConvColor,
      minimum height=3cm,
      minimum width=2.2cm,
      rounded corners=3pt,
      text width=2cm,
      align=center] (C) at (8,0) {
        \small\textbf{Capa C}\\[0.2cm]
        \footnotesize Conv. Híbrida\\
        \footnotesize AdaptiveLIF\\[0.1cm]
        \tiny N = 100 neuronas\\
        \tiny kernel: Gaussian\\
        \tiny size $\in$ [3,9], $\sigma$ $\in$ [0.5,3]\\
        \tiny balance exc/inh
};

% === Bloque de Fusión y Decisión ===
\node[draw=black,
      fill=FusionColor,
      minimum height=2cm,
      minimum width=1.8cm,
      rounded corners=3pt,
      text width=1.6cm,
      align=center] (Fusion) at (11,0) {
        \small\textbf{Fusión}\\[0.2cm]
        \footnotesize Modos:\\
        \tiny $\bullet$ direct\\
        \tiny $\bullet$ weighted\_sum\\
        \tiny $\bullet$ max\\[0.1cm]
        \footnotesize Optimización:\\
        \tiny max(F1\_B, F1\_C)
};

% === Conexiones principales ===
\draw[conexion] (A.east) -- (B.west) node[midway,above] {\footnotesize A$\rightarrow$B};
\draw[conexion] (B.east) -- (C.west) node[midway,above] {\footnotesize B$\rightarrow$C};
\draw[conexion] (C.east) -- (Fusion.west) node[midway,above] {\footnotesize C$\rightarrow$Fusión};

% === Conexión de retroalimentación B->B (recurrente) ===
\draw[conexion, dashed, bend left=60] (B.north east) to node[above] {\tiny recurrente} (B.north west);

% === Conexión dual B->Fusión (para weighted_sum y max) ===
\draw[conexion, bend right=20, opacity=0.6] (B.south) to node[below] {\tiny B$\rightarrow$Fusión} (Fusion.south west);

% === Etiquetas de dominio ===
\node[above=0.5cm of A] {\footnotesize \textbf{Dominio de Codificación}};
\node[above=0.5cm of B] {\footnotesize \textbf{Dominio SNN}};
\node[above=0.5cm of C] {\footnotesize \textbf{Dominio CNN}};
\node[above=0.5cm of Fusion] {\footnotesize \textbf{Decisión}};

% === Flujo de datos (flechas de información) ===
\node[below=0.3cm of A] {\tiny Series temporales};
\node[below=0.3cm of B] {\tiny Spikes B};
\node[below=0.3cm of C] {\tiny Spikes C};
\node[below=0.3cm of Fusion] {\tiny Anomalía/Normal};

% === Líneas divisorias entre dominios (opcionales) ===
\draw[dashed, gray, opacity=0.5] (2.2,-2) -- (2.2,2.5);
\draw[dashed, gray, opacity=0.5] (6.2,-2) -- (6.2,2.5);
\draw[dashed, gray, opacity=0.5] (9.8,-2) -- (9.8,2.5);

\end{tikzpicture}

\caption{Arquitectura del modelo híbrido SNN-CNN. El modelo procesa series temporales mediante tres dominios principales: (1) codificación por cuantiles, (2) procesamiento SNN con neuronas LIF y aprendizaje STDP, (3) refinamiento convolucional con kernels adaptativos, y (4) fusión dual para optimización de detección de anomalías. Las conexiones recurrentes en la capa B y la retroalimentación entre dominios permiten el procesamiento jerárquico de patrones temporales complejos.}
\label{fig:arquitectura-hibrida}

\end{figure}

\begin{figure}[h]
\centering
\begin{tikzpicture}[scale=0.8]

% Definiciones de colores específicas para el diagrama (manteniendo compatibilidad)
\definecolor{EntradaColor}{rgb}{1,0.6,0.2}
\definecolor{SNNColor}{rgb}{0.2,0.6,1}
\definecolor{ConvColor}{rgb}{0.8,0.3,0.8}
\definecolor{FusionColor}{rgb}{0.4,0.8,0.4}

% Estilo de conexión personalizado (corregido para evitar conflictos)
\tikzset{
    conexion/.style={ultra thick, ->, >=stealth, opacity=0.8}
}

% === Caja A - Entrada (Codificación de cuantiles) ===
\node[draw=black,
      fill=EntradaColor,
      minimum height=3cm,
      minimum width=1.5cm,
      rounded corners=3pt,
      text width=1.3cm,
      align=center] (A) at (0,0) {
        \small\textbf{Entrada}\\[0.2cm]
        \footnotesize Codificación\\
        cuantiles\\[0.1cm]
        \tiny N = len(cuantiles)-1\\
        \tiny $\sim$64 neuronas
};

% === Caja B - Capa SNN (LIF + STDP) ===
\node[draw=black,
      fill=SNNColor,
      minimum height=3.5cm,
      minimum width=2cm,
      rounded corners=3pt,
      text width=1.8cm,
      align=center] (B) at (4,0) {
        \small\textbf{Capa B}\\[0.2cm]
        \footnotesize SNN Base\\
        \footnotesize LIF + Recurrente\\[0.1cm]
        \tiny N = 100 neuronas\\
        \tiny threshold $\in$ [-65,-50]\\
        \tiny decay $\in$ [80,150]\\
        \tiny STDP: nu1, nu2 $\in$ [-0.5,0.5]
};

% === Caja C - Capa Convolucional Híbrida ===
\node[draw=black,
      fill=ConvColor,
      minimum height=3cm,
      minimum width=2.2cm,
      rounded corners=3pt,
      text width=2cm,
      align=center] (C) at (8,0) {
        \small\textbf{Capa C}\\[0.2cm]
        \footnotesize Conv. Híbrida\\
        \footnotesize AdaptiveLIF\\[0.1cm]
        \tiny N = 100 neuronas\\
        \tiny kernel: Gaussian\\
        \tiny size $\in$ [3,9], $\sigma$ $\in$ [0.5,3]\\
        \tiny balance exc/inh
};

% === Bloque de Fusión y Decisión ===
\node[draw=black,
      fill=FusionColor,
      minimum height=2cm,
      minimum width=1.8cm,
      rounded corners=3pt,
      text width=1.6cm,
      align=center] (Fusion) at (11,0) {
        \small\textbf{Fusión}\\[0.2cm]
        \footnotesize Modos:\\
        \tiny $\bullet$ direct\\
        \tiny $\bullet$ weighted\_sum\\
        \tiny $\bullet$ max\\[0.1cm]
        \footnotesize Optimización:\\
        \tiny max(F1\_B, F1\_C)
};

% === Conexiones principales ===
\draw[conexion] (A.east) -- (B.west) node[midway,above] {\footnotesize A$\rightarrow$B};
\draw[conexion] (B.east) -- (C.west) node[midway,above] {\footnotesize B$\rightarrow$C};
\draw[conexion] (C.east) -- (Fusion.west) node[midway,above] {\footnotesize C$\rightarrow$Fusión};

% === Conexión de retroalimentación B->B (recurrente) ===
\draw[conexion, dashed, bend left=60] (B.north east) to node[above] {\tiny recurrente} (B.north west);

% === Conexión dual B->Fusión (para weighted_sum y max) ===
\draw[conexion, bend right=20, opacity=0.6] (B.south) to node[below] {\tiny B$\rightarrow$Fusión} (Fusion.south west);

% === Etiquetas de dominio ===
\node[above=0.5cm of A] {\footnotesize \textbf{Dominio de Codificación}};
\node[above=0.5cm of B] {\footnotesize \textbf{Dominio SNN}};
\node[above=0.5cm of C] {\footnotesize \textbf{Dominio CNN}};
\node[above=0.5cm of Fusion] {\footnotesize \textbf{Decisión}};

% === Flujo de datos (flechas de información) ===
\node[below=0.3cm of A] {\tiny Series temporales};
\node[below=0.3cm of B] {\tiny Spikes B};
\node[below=0.3cm of C] {\tiny Spikes C};
\node[below=0.3cm of Fusion] {\tiny Anomalía/Normal};

% === Líneas divisorias entre dominios (opcionales) ===
\draw[dashed, gray, opacity=0.5] (2.2,-2) -- (2.2,2.5);
\draw[dashed, gray, opacity=0.5] (6.2,-2) -- (6.2,2.5);
\draw[dashed, gray, opacity=0.5] (9.8,-2) -- (9.8,2.5);

\end{tikzpicture}

\caption{Arquitectura del modelo híbrido SNN-CNN. El modelo procesa series temporales mediante tres dominios principales: (1) codificación por cuantiles, (2) procesamiento SNN con neuronas LIF y aprendizaje STDP, (3) refinamiento convolucional con kernels adaptativos, y (4) fusión dual para optimización de detección de anomalías. Las conexiones recurrentes en la capa B y la retroalimentación entre dominios permiten el procesamiento jerárquico de patrones temporales complejos.}
\label{fig:arquitectura-hibrida}

\end{figure}

\begin{figure}[h]
\centering
\begin{tikzpicture}[scale=0.8]

% Definiciones de colores específicas para el diagrama (manteniendo compatibilidad)
\definecolor{EntradaColor}{rgb}{1,0.6,0.2}
\definecolor{SNNColor}{rgb}{0.2,0.6,1}
\definecolor{ConvColor}{rgb}{0.8,0.3,0.8}
\definecolor{FusionColor}{rgb}{0.4,0.8,0.4}

% Estilo de conexión personalizado (corregido para evitar conflictos)
\tikzset{
    conexion/.style={ultra thick, ->, >=stealth, opacity=0.8}
}

% === Caja A - Entrada (Codificación de cuantiles) ===
\node[draw=black,
      fill=EntradaColor,
      minimum height=3cm,
      minimum width=1.5cm,
      rounded corners=3pt,
      text width=1.3cm,
      align=center] (A) at (0,0) {
        \small\textbf{Entrada}\\[0.2cm]
        \footnotesize Codificación\\
        cuantiles\\[0.1cm]
        \tiny N = len(cuantiles)-1\\
        \tiny $\sim$64 neuronas
};

% === Caja B - Capa SNN (LIF + STDP) ===
\node[draw=black,
      fill=SNNColor,
      minimum height=3.5cm,
      minimum width=2cm,
      rounded corners=3pt,
      text width=1.8cm,
      align=center] (B) at (4,0) {
        \small\textbf{Capa B}\\[0.2cm]
        \footnotesize SNN Base\\
        \footnotesize LIF + Recurrente\\[0.1cm]
        \tiny N = 100 neuronas\\
        \tiny threshold $\in$ [-65,-50]\\
        \tiny decay $\in$ [80,150]\\
        \tiny STDP: nu1, nu2 $\in$ [-0.5,0.5]
};

% === Caja C - Capa Convolucional Híbrida ===
\node[draw=black,
      fill=ConvColor,
      minimum height=3cm,
      minimum width=2.2cm,
      rounded corners=3pt,
      text width=2cm,
      align=center] (C) at (8,0) {
        \small\textbf{Capa C}\\[0.2cm]
        \footnotesize Conv. Híbrida\\
        \footnotesize AdaptiveLIF\\[0.1cm]
        \tiny N = 100 neuronas\\
        \tiny kernel: Gaussian\\
        \tiny size $\in$ [3,9], $\sigma$ $\in$ [0.5,3]\\
        \tiny balance exc/inh
};

% === Bloque de Fusión y Decisión ===
\node[draw=black,
      fill=FusionColor,
      minimum height=2cm,
      minimum width=1.8cm,
      rounded corners=3pt,
      text width=1.6cm,
      align=center] (Fusion) at (11,0) {
        \small\textbf{Fusión}\\[0.2cm]
        \footnotesize Modos:\\
        \tiny $\bullet$ direct\\
        \tiny $\bullet$ weighted\_sum\\
        \tiny $\bullet$ max\\[0.1cm]
        \footnotesize Optimización:\\
        \tiny max(F1\_B, F1\_C)
};

% === Conexiones principales ===
\draw[conexion] (A.east) -- (B.west) node[midway,above] {\footnotesize A$\rightarrow$B};
\draw[conexion] (B.east) -- (C.west) node[midway,above] {\footnotesize B$\rightarrow$C};
\draw[conexion] (C.east) -- (Fusion.west) node[midway,above] {\footnotesize C$\rightarrow$Fusión};

% === Conexión de retroalimentación B->B (recurrente) ===
\draw[conexion, dashed, bend left=60] (B.north east) to node[above] {\tiny recurrente} (B.north west);

% === Conexión dual B->Fusión (para weighted_sum y max) ===
\draw[conexion, bend right=20, opacity=0.6] (B.south) to node[below] {\tiny B$\rightarrow$Fusión} (Fusion.south west);

% === Etiquetas de dominio ===
\node[above=0.5cm of A] {\footnotesize \textbf{Dominio de Codificación}};
\node[above=0.5cm of B] {\footnotesize \textbf{Dominio SNN}};
\node[above=0.5cm of C] {\footnotesize \textbf{Dominio CNN}};
\node[above=0.5cm of Fusion] {\footnotesize \textbf{Decisión}};

% === Flujo de datos (flechas de información) ===
\node[below=0.3cm of A] {\tiny Series temporales};
\node[below=0.3cm of B] {\tiny Spikes B};
\node[below=0.3cm of C] {\tiny Spikes C};
\node[below=0.3cm of Fusion] {\tiny Anomalía/Normal};

% === Líneas divisorias entre dominios (opcionales) ===
\draw[dashed, gray, opacity=0.5] (2.2,-2) -- (2.2,2.5);
\draw[dashed, gray, opacity=0.5] (6.2,-2) -- (6.2,2.5);
\draw[dashed, gray, opacity=0.5] (9.8,-2) -- (9.8,2.5);

\end{tikzpicture}

\caption{Arquitectura del modelo híbrido SNN-CNN. El modelo procesa series temporales mediante tres dominios principales: (1) codificación por cuantiles, (2) procesamiento SNN con neuronas LIF y aprendizaje STDP, (3) refinamiento convolucional con kernels adaptativos, y (4) fusión dual para optimización de detección de anomalías. Las conexiones recurrentes en la capa B y la retroalimentación entre dominios permiten el procesamiento jerárquico de patrones temporales complejos.}
\label{fig:arquitectura-hibrida}

\end{figure}

\begin{figure}[h]
\centering
\begin{tikzpicture}[scale=0.8]

% Definiciones de colores específicas para el diagrama (manteniendo compatibilidad)
\definecolor{EntradaColor}{rgb}{1,0.6,0.2}
\definecolor{SNNColor}{rgb}{0.2,0.6,1}
\definecolor{ConvColor}{rgb}{0.8,0.3,0.8}
\definecolor{FusionColor}{rgb}{0.4,0.8,0.4}

% Estilo de conexión personalizado (corregido para evitar conflictos)
\tikzset{
    conexion/.style={ultra thick, ->, >=stealth, opacity=0.8}
}

% === Caja A - Entrada (Codificación de cuantiles) ===
\node[draw=black,
      fill=EntradaColor,
      minimum height=3cm,
      minimum width=1.5cm,
      rounded corners=3pt,
      text width=1.3cm,
      align=center] (A) at (0,0) {
        \small\textbf{Entrada}\\[0.2cm]
        \footnotesize Codificación\\
        cuantiles\\[0.1cm]
        \tiny N = len(cuantiles)-1\\
        \tiny $\sim$64 neuronas
};

% === Caja B - Capa SNN (LIF + STDP) ===
\node[draw=black,
      fill=SNNColor,
      minimum height=3.5cm,
      minimum width=2cm,
      rounded corners=3pt,
      text width=1.8cm,
      align=center] (B) at (4,0) {
        \small\textbf{Capa B}\\[0.2cm]
        \footnotesize SNN Base\\
        \footnotesize LIF + Recurrente\\[0.1cm]
        \tiny N = 100 neuronas\\
        \tiny threshold $\in$ [-65,-50]\\
        \tiny decay $\in$ [80,150]\\
        \tiny STDP: nu1, nu2 $\in$ [-0.5,0.5]
};

% === Caja C - Capa Convolucional Híbrida ===
\node[draw=black,
      fill=ConvColor,
      minimum height=3cm,
      minimum width=2.2cm,
      rounded corners=3pt,
      text width=2cm,
      align=center] (C) at (8,0) {
        \small\textbf{Capa C}\\[0.2cm]
        \footnotesize Conv. Híbrida\\
        \footnotesize AdaptiveLIF\\[0.1cm]
        \tiny N = 100 neuronas\\
        \tiny kernel: Gaussian\\
        \tiny size $\in$ [3,9], $\sigma$ $\in$ [0.5,3]\\
        \tiny balance exc/inh
};

% === Bloque de Fusión y Decisión ===
\node[draw=black,
      fill=FusionColor,
      minimum height=2cm,
      minimum width=1.8cm,
      rounded corners=3pt,
      text width=1.6cm,
      align=center] (Fusion) at (11,0) {
        \small\textbf{Fusión}\\[0.2cm]
        \footnotesize Modos:\\
        \tiny $\bullet$ direct\\
        \tiny $\bullet$ weighted\_sum\\
        \tiny $\bullet$ max\\[0.1cm]
        \footnotesize Optimización:\\
        \tiny max(F1\_B, F1\_C)
};

% === Conexiones principales ===
\draw[conexion] (A.east) -- (B.west) node[midway,above] {\footnotesize A$\rightarrow$B};
\draw[conexion] (B.east) -- (C.west) node[midway,above] {\footnotesize B$\rightarrow$C};
\draw[conexion] (C.east) -- (Fusion.west) node[midway,above] {\footnotesize C$\rightarrow$Fusión};

% === Conexión de retroalimentación B->B (recurrente) ===
\draw[conexion, dashed, bend left=60] (B.north east) to node[above] {\tiny recurrente} (B.north west);

% === Conexión dual B->Fusión (para weighted_sum y max) ===
\draw[conexion, bend right=20, opacity=0.6] (B.south) to node[below] {\tiny B$\rightarrow$Fusión} (Fusion.south west);

% === Etiquetas de dominio ===
\node[above=0.5cm of A] {\footnotesize \textbf{Dominio de Codificación}};
\node[above=0.5cm of B] {\footnotesize \textbf{Dominio SNN}};
\node[above=0.5cm of C] {\footnotesize \textbf{Dominio CNN}};
\node[above=0.5cm of Fusion] {\footnotesize \textbf{Decisión}};

% === Flujo de datos (flechas de información) ===
\node[below=0.3cm of A] {\tiny Series temporales};
\node[below=0.3cm of B] {\tiny Spikes B};
\node[below=0.3cm of C] {\tiny Spikes C};
\node[below=0.3cm of Fusion] {\tiny Anomalía/Normal};

% === Líneas divisorias entre dominios (opcionales) ===
\draw[dashed, gray, opacity=0.5] (2.2,-2) -- (2.2,2.5);
\draw[dashed, gray, opacity=0.5] (6.2,-2) -- (6.2,2.5);
\draw[dashed, gray, opacity=0.5] (9.8,-2) -- (9.8,2.5);

\end{tikzpicture}

\caption{Arquitectura del modelo híbrido SNN-CNN. El modelo procesa series temporales mediante tres dominios principales: (1) codificación por cuantiles, (2) procesamiento SNN con neuronas LIF y aprendizaje STDP, (3) refinamiento convolucional con kernels adaptativos, y (4) fusión dual para optimización de detección de anomalías. Las conexiones recurrentes en la capa B y la retroalimentación entre dominios permiten el procesamiento jerárquico de patrones temporales complejos.}
\label{fig:arquitectura-hibrida}

\end{figure}