%\begin{center}
%{\Large \bfseries Introducción}
%\vspace{2.5cm}
%\end{center}
\section{Motivación}

En el mundo actual, donde los datos crecen de forma exponencial, detectar comportamientos anómalos en series temporales se ha vuelto algo crítico para muchos sectores. Ya sea en la monitorización de redes informáticas o en el control de sistemas industriales, identificar patrones irregulares a tiempo puede prevenir fallos, optimizar recursos y garantizar la seguridad de infraestructuras críticas.

Las anomalías en series temporales son desviaciones significativas de lo que consideraríamos comportamiento normal o esperado en un sistema. Pueden aparecer como picos inusuales, cambios bruscos en la tendencia, o alteraciones en cómo se repiten los datos periódicamente. Detectar estas irregularidades de forma efectiva es complicado, sobre todo cuando trabajamos con sistemas complejos que generan grandes cantidades de datos en tiempo real. 

Durante las últimas décadas se han desarrollado varios métodos para abordar este problema, desde técnicas estadísticas tradicionales hasta algoritmos complejos de aprendizaje automático. Más recientemente, los modelos de aprendizaje profundo como autoencoders y redes neuronales recurrentes han demostrado un buen rendimiento en este campo, gracias a que pueden capturar relaciones no lineales y patrones temporales en los datos \cite{cherdo_time_2023,kshirasagar_auditory_2024}. Pero estos modelos convencionales tienen sus limitaciones, especialmente en cuanto a eficiencia computacional y consumo energético—algo crítico en dispositivos con recursos limitados o cuando necesitas procesamiento en tiempo real \cite{kshirasagar_auditory_2024,plummer_2d_2025}. 

Las SNNs aparecen como un enfoque interesante que intenta imitar mejor cómo funciona el cerebro biológico. Ofrecen ventajas en eficiencia energética y procesamiento temporal inherente comparado con los modelos convencionales de deep learning \cite{lv_efficient_2024,yusob_anomaly_2018,basler_unsupervised_2022}. Este trabajo se enmarca en la línea de investigación que abrieron estudios pioneros que demostraron que las SNNs pueden funcionar para detección de anomalías mediante mecanismos de inhibición sináptica y codificación temporal \cite{cherdo_time_2023,yusob_anomaly_2018,basler_unsupervised_2022}. Proponemos mejoras arquitecturales y de optimización que superan limitaciones identificadas en implementaciones previas. 

\section{Objetivos}

El objetivo general del proyecto es el desarrollo y optimización de sistemas de detección de anomalías basados en SNNs

\begin{enumerate}
    \item Revisión bibliográfica de modelos de SNNs y su aplicación en detección de anomalías y mantenimiento predictivo.
    \item Diseño y desarrollo de arquitecturas de SNNs optimizadas para la detección de anomalías en datos de sensores industriales.
    \item Implementación de modelos de SNNs sostenibles que minimicen el consumo energético sin sacrificar la precisión.
    \item Validación de los modelos desarrollados mediante experimentación en conjuntos de datos reales de mantenimiento predictivo.
    \item Comparación de la eficiencia energética y el rendimiento predictivo frente a modelos tradicionales de detección de anomalías.
%  \item Revisión bibliográfica de modelos de Redes Neuronales de Impulsos
% (SNNs) y su aplicación en detección de anomalías y mantenimiento
% predictivo.
%  \item Diseño y desarrollo de arquitecturas de SNNs optimizadas para la
% detección de anomalías en datos de sensores industriales.
%  \item Implementación de modelos de SNNs sostenibles que minimicen el
% consumo energético sin sacrificar la precisión.
%  \item Validación de los modelos desarrollados mediante experimentación en
% conjuntos de datos reales de mantenimiento predictivo.
%  \item Comparación de la eficiencia energética y el rendimiento predictivo
% frente a modelos tradicionales de detección de anomalías.
\end{enumerate}


\section{Estructura de la memoria}


A continuación se describe la estructura de esta memoria. Consta de 5 capítulos:

\begin{itemize}
    \item\textbf{Capítulo 1}: Este primer capítulo presenta la motivación del proyecto, explicando por qué es tan importante detectar anomalías en series temporales para distintos sectores industriales. Se establecen los objetivos específicos: diseñar arquitecturas SNNs optimizadas e implementar modelos sostenibles que reduzcan el consumo energético. También se incluye la planificación temporal del proyecto con estimaciones de costes y duración de las tareas principales.
    
    \item\textbf{Capítulo 2}: Aquí se hace un análisis de la evolución histórica de los métodos de detección de anomalías, desde técnicas estadísticas clásicas hasta los enfoques más recientes de aprendizaje profundo. Se examinan los fundamentos estadísticos y enfoques tradicionales, las aplicaciones del aprendizaje profundo en detección de anomalías, y métodos de optimización de hiperparámetros. Se dedica especial atención a las redes neuronales de impulsos y la computación neuromórfica, el desarrollo de benchmarks especializados, y la comparación de eficiencia energética entre diferentes paradigmas computacionales.
    
    \item\textbf{Capítulo 3}: En este capítulo se define el alcance específico del proyecto, las hipótesis de partida y las restricciones identificadas. Se describe en detalle la solución propuesta, incluyendo la justificación para usar BindsNET como framework de desarrollo y los desafíos técnicos encontrados durante la configuración del entorno. El capítulo profundiza en la arquitectura del nuevo modelo híbrido SNN-CNN, explicando qué innovaciones introducimos respecto a la implementación original, y describe los algoritmos de preprocesamiento desarrollados para los datasets IOPS y CalIt2.
    
    \item\textbf{Capítulo 4}: Este capítulo presenta la evaluación experimental del modelo propuesto. Se describen los escenarios experimentales implementados, incluyendo la configuración de hardware y software, los datasets utilizados y las estrategias de particionado temporal. Se detalla la metodología de evaluación, con especial énfasis en las métricas de calidad (precisión, recall, F1-score) y eficiencia computacional. El análisis de resultados incluye comparaciones entre la SNN original, el modelo híbrido propuesto y baselines de la librería TSFEDL, la importancia de los hiperparámetros, y evaluaciones de escalabilidad.
    
    \item\textbf{Capítulo 5}: El último capítulo sintetiza los principales hallazgos del trabajo, evaluando si se cumplieron los objetivos planteados y cuál fue el impacto de las contribuciones realizadas. Se discuten las limitaciones identificadas durante el desarrollo y se proponen direcciones para trabajos futuros que puedan extender y mejorar los métodos desarrollados.
\end{itemize}

% En el desarrollo de este capítulo (Capítulo 1) hasta ahora se ha presentado la introducción de este trabajo, lo que se abordará, la motivación y objetivos por los que se rige el mismo.

% En el capítulo a continuación (Capítulo 3) se detallan los fundamentos y antecedentes en los que se comienza el estudio de la correferencia de entidades, su definición, sistemas de correferencia actuales de amplio uso tanto en inglés como en español, así como se presentan las métricas habituales que se usan en la evaluación de estos sistemas. El objetivo de este capítulo es dar a conocer conceptos básicos y precedentes que dan lugar al problema que se aborda en este trabajo.

% En el Capítulo 4 se detallan las especificaciones del trabajo, el alcance del TFM, la hipótesis de partida y las restricciones que se presentan en la solución de este problema. Además, se presenta un estudio de alternativas existentes y la viabilidad de estas. Este capítulo también tiene como objetivo presentar brevemente cómo se abordará la solución que se propone, así como la metodología, planificación y presupuesto para su desarrollo.

% En el Capítulo 5, se presentan los materiales y métodos que se utilizan en este trabajo. Para el flujo de la detección de correferencia planteado se presentan los datos de prueba, los modelos utilizados para la traducción de los textos de prueba, los modelos de correferencia utilizados y los modelos de alineamiento para volver al idioma original. En este capítulo, además de conocerse el flujo de trabajo para la solución que se plantea, se conocen las justificaciones de las elecciones de modelos sobre otros antes expuestos.

% A continuación, el Capítulo 5, presenta la descripción y resultados de los experimentos realizados para probar la solución que se plantea en este trabajo.

% Este documento finaliza con las conclusiones expuestas en el Capítulo 6, junto a las recomendaciones de trabajos futuros para mejorar los métodos expuestos.
