\section{Redes neuronales de impulsos y computación neuromórfica}
\subsection{Inspiración biológica y eficiencia energética}

Las redes neuronales de impulsos representan un cambio de paradigma hacia modelos de computación biológicamente plausibles que enfatizan el procesamiento basado en eventos, tiempos de respuesta bajos y una eficiencia energética excepcional \cite{maass_networks_1997}. El desarrollo de reglas de aprendizaje de plasticidad dependiente del tiempo de impulsos (\textbf{STDP}), con raíces históricas que se extienden a conceptos filosóficos de Aristóteles y Locke, culminó en implementaciones prácticas que combinan simplicidad elegante con plausibilidad biológica y poder computacional \cite{markram_history_2011}.

Las implementaciones de hardware neuromórfico alrededor de 2011 permitieron el despliegue práctico de sistemas de detección de anomalías basados en SNNs con una eficiencia energética sin precedentes \cite{indiveri_neuromorphic_2011}.

\subsection{Aplicaciones modernas y arquitecturas híbridas}

Los desarrollos recientes en aplicaciones de SNN para la detección de anomalías se han centrado en la detección robusta de anomalías de audio, donde los modelos multivariados de series temporales robustos a valores atípicos detectan sonidos anómalos previamente no vistos basados en datos de entrenamiento ruidosos \cite{lee_robust_2022}. Estos enfoques utilizan arquitecturas novedosas de redes neuronales profundas que aprenden dinámicas temporales a múltiples resoluciones mientras mantienen robustez frente a contaminaciones en el conjunto de datos de entrenamiento. Las dinámicas temporales se modelan utilizando capas recurrentes aumentadas con mecanismos de atención construidos sobre capas convolucionales, lo que permite la extracción de características a múltiples resoluciones.

La aparición de arquitecturas híbridas \textbf{SNN-CNN} alrededor de 2024 representa el avance más reciente, combinando la eficiencia energética de las redes de impulsos con las capacidades de reconocimiento de patrones de las redes neuronales convolucionales \cite{sanaullah_hybrid_2024}. Estos sistemas ofrecen una precisión mejorada mientras mantienen las ventajas de bajo consumo de energía de la computación neuromórfica, lo que los hace particularmente adecuados para aplicaciones donde la vida útil de la batería y los recursos computacionales están limitados \cite{liu_sparsity-aware_2024}.