\section{Aprendizaje profundo en la detección de anomalías}
\subsection{Fundamentos de redes neuronales y métodos basados en reconstrucción}

La revolución del aprendizaje profundo transformó la capacidad de detectar anomalías mediante arquitecturas neuronales capaces de aprender patrones temporales complejos. Los \textbf{autoencoders profundos} surgieron como paradigma dominante alrededor de 2006, utilizando el error de reconstrucción como principio fundamental para identificar anomalías . Estas redes aprenden a comprimir patrones normales en representaciones de menor dimensión y luego los reconstruyen, con el supuesto de que los patrones anómalos exhibirán mayores errores de reconstrucción debido a su desviación de los comportamientos normales aprendidos \cite{hinton_reducing_2006}.

Las redes de memoria de corto y largo plazo (\textbf{LSTM}) fue propuesta en 1997, en aquel momento existían limitaciones críticas en el modelado temporal al proporcionar mecanismos para capturar tanto dependencias a corto como a largo plazo en datos de series temporales \cite{hochreiter_long_1997}. Las implementaciones modernas a partir del 2010 a menudo integran redes LSTM con mecanismos de atención multi-cabeza y capas de red completamente conectadas para mejorar la precisión de predicción y las capacidades de detección de anomalías. Estas arquitecturas híbridas han demostrado ser particularmente efectivas en el análisis de transacciones blockchain, donde modelan con éxito la naturaleza dinámica, no lineal y variable en el tiempo de los patrones de transacción \cite{xia_novel_2024}.

\subsection{Arquitecturas avanzadas de aprendizaje profundo}

El desarrollo de los autoencoders variacionales (\textbf{VAEs}) alrededor de 2014 introdujo enfoques probabilísticos en la detección de anomalías, permitiendo la cuantificación de la incertidumbre y un manejo más robusto de distribuciones de datos complejas \cite{kingma_auto-encoding_2022-1}. Trabajos recientes han demostrado la efectividad de las perspectivas variacionales espacio-temporales jerárquicas en la detección de anomalías en series temporales multivariadas, modelando explícitamente variables estocásticas temporales y variables de relación de gráficos latentes dentro de marcos gráficos unificados \cite{zhang_rethinking_2024,yang_h-vgrae_2020}.

En 2018, ganan prominencia los enfoques de aprendizaje autosupervisado, particularmente a través de la aplicación de técnicas de codificación predictiva contrastiva (\textbf{CPC}). En este estudio se busca transformar la identificación de anomalías en problemas de identificación de pares positivos \cite{noauthor_180703748_nodate}. El esquema CPC de salto de paso representa un avance notable, ajustando la distancia entre ventanas de historial y puntos de detección para mejorar la construcción de pares positivos y el rendimiento de detección \cite{zhang_skip-step_2024}.