\section{Conclusiones}

La comparación entre las ANN, las SNN y las SNN optimizadas con Optuna revela importantes compromisos que deberían guiar las decisiones de implementación para aplicaciones de detección de anomalías. Mientras que las ANN ofrecen madurez y un rendimiento establecido, las SNN proporcionan ventajas significativas en términos de eficiencia energética, y la adición de la optimización con Optuna puede cerrar la brecha de precisión al mismo tiempo que mejora aún más la eficiencia.

A medida que el hardware neuromórfico continúa evolucionando, las ventajas de eficiencia energética de las SNN probablemente se volverán cada vez más relevantes, particularmente para aplicaciones de edge computing y el Internet de las cosas (IoT), donde las restricciones de energía son significativas. Las tendencias acutales sugieren que las SNN optimizadas con Optuna representan una dirección prometedora para futuros sistemas de detección de anomalías que requieren tanto alta precisión como eficiencia energética.
