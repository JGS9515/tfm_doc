\chapter{Añadir una Tabla}
Para añadir una tabla en latex debes incluir el siguiente código:
%Con {verbatim} añadimos textualmente lo que escribimos, sin que latex lo procese
\begin{verbatim}
\begin{table}[1*]
  \centering
  \begin{tabular}{2*}
\hline \hline
3* & 4*\\
\hline
5* & 6*\\
7* & 8*\\
\hline \hline
  \end{tabular}
  \caption{9*}
  \label{10*}
\end{table}
\end{verbatim}
1* -> sustituir el 1 por:\\
\\
h:	Establece la posición del elemento flotante «aquí». Ésto es, aproximadamente en el mismo punto donde aparece en el código (sin embargo, no siempre es exacto el posicionamiento)\\
\\
t:	Inserta la figura al inicio de la página.\\
\\
b:	Inserta la figura al final de la página.\\
\\
p:	Inserta los elementos flotantes en una página por separado, que sólo contiene figuras.\\
\\
!:	Sobrescribe los parámetros que LATEX usa para determinar una «buena» posición para la imagen.\\
\\
Nota: Se pueden combinar varias opciones.\\
\\
2* -> Sustituir el 2 por:\\
\\
l,c,r. Debemos añadir tantas letras como columnas tengamos, significando l alineación a la izquierda, c alineación centrada y r alineación a la derecha.\\
\\
3*,4* -> Sustituir el 3 y 4 por:\\
\\
Título columna 1, título columna 2. \\
\\
5*, 6*, 7*, 8* -> Sustituir el 5, 6, 7 y 8 por:\\
\\
Dato [1][1], dato[1][2], dato[2][1], dato[2][2]\\
\\
9* -> Sustituir el 9 por:\\
\\
Texto descriptivo de la tabla\\
\\
10* -> Sustituir el 10 por:\\
\\
La etiqueta de la tabla, sera el ID de la tabla que nos servirá para referenciarla.\\
\\
%Ejemplo de inclusión de una tabla, este formato es un ejemplo se puede modificar el aspecto de la tabla a conveniencia.
\begin{table}[!h]
  \centering
  \begin{tabular}{lc}
  %Doble linea de la parte superior de la tabla.
\hline \hline
Tit col 1 & tit col 2*\\
%Linea de separación entre tittulos de columna y datos.
\hline
1,1 & 1,2\\
2,1 & 2,2\\
%Doble linea final de tabla.
\hline \hline
  \end{tabular}
  \caption{Tabla de ejemplo}
  \label{TabEjem}
\end{table}\\
Para referenciar una tabla o cualquier elemento con un label usamos el comando \begin{verbatim}\ref{nombre_etiqueta}\end{verbatim}. De esta forma la tabla \textbf{ejemplo }se referencia como \ref{TabEjem}.\\

Para citar una referencia bibliográfica se hace de la siguiente forma: \begin{verbatim}   \cite{nombre_etiqueta}\end{verbatim} Por ejemplo para citar el libro de Knuth, que puede verse al final de este documento, se escribiría: \begin{verbatim}\cite{Discriminador1}\end{verbatim} y se mostraría \cite{Discriminador1}

















