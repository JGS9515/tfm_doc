\chapter{Resultados}
\section{Descripción de los escenarios experimentales}
Breve recordatorio de

	•	Datasets utilizados y su finalidad (entrenamiento, validación, prueba).
    
	•	Configuraciones de hardware (CPU / GPU, memoria, versión de librerías).
    
	•	Conjuntos de hiperparámetros explorados y estrategia de búsqueda (p. ej. Optuna–TPE, grid search…).
		Tabla 6-1. Resumen de configuraciones ensayadas.

\section{Metodología de evaluación}
    1.	Métricas de calidad.
	
    2.	Métricas de eficiencia (tiempo de entrenamiento, tiempo de inferencia)
% \section{Experimentos realizados}
% Que metrica se utiliza y por que
% \section{Datos obtenidos}
% tablas con metricas
\section{Análasis de Resultados}
presentar resultados
detecciones de fallos, comparando con otros modelos
