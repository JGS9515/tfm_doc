\section{Descripción de los escenarios experimentales}

En este capítulo se presentan los escenarios, configuraciones y protocolos empleados para evaluar los modelos de detección de anomalías. Se comparan:
\begin{itemize}
    \item \textbf{SNN original (A--B)}: implementación base empleada en el trabajo previo.
    \item \textbf{SNN híbrida (A--B--C)}: modelo propuesto con capa convolucional y búsqueda de hiperparámetros (script \texttt{javi/ejecutar\_experimento\_javi.py}).
    \item \textbf{Baselines TSFEDL}: modelos seleccionados del repositorio \texttt{@JGS9515/compare\_to\_TSFEDL}\footnote{\url{https://github.com/JGS9515/compare_to_TSFEDL}}.
\end{itemize}

\subsection{Datasets y particionado}
% TODO: Rellenar con datos reales de cada dataset (tamaños, % positivos, etc.)
Se emplearon datasets públicos y/o internos con etiquetas binarias (0: normal, 1: anomalía). Para cada uno:
\begin{itemize}
    \item \textbf{IOPS}: KPI de servicios (Input/Output Operations Per Second). 
    \begin{itemize}
        \item Observaciones: \textit{[TODO: N]}.
        \item Frecuencia de muestreo: \textit{[TODO]}.
        \item Variables: \texttt{value}, \texttt{label}.
        \item Porcentaje de positivos: \textit{[TODO:\%]}.
        \item Particionado: 50/50 temporal (primera mitad entrenamiento, segunda mitad prueba), preservando orden temporal para evitar fuga de información.
    \end{itemize}
    \item \textbf{CalIt2}: flujos de entrada/salida en el edificio CalIt2 (UCI).
    \begin{itemize}
        \item Observaciones: 10.080 (15 semanas, 48 intervalos/día).
        \item Frecuencia de muestreo: 30 minutos.
        \item Variables: \texttt{value} (univariado por flujo), \texttt{label}.
        \item Porcentaje de positivos: \textit{[TODO:\%]}.
        \item Particionado: 50/50 temporal. 
    \end{itemize}
\end{itemize}

Preprocesado común:
\begin{itemize}
    \item Tipado de columnas: \texttt{value} en \texttt{float64} y \texttt{label} en \texttt{Int64}.
    \item Expansión de etiquetas en entrenamiento (\texttt{expansion = 100}) para mitigar desbalanceo temporal.
    \item Cálculo de cuantiles sobre \textbf{train} únicamente: rango extendido con \texttt{a = 0{,}1} y resolución \texttt{r = 0{,}05}; determina \texttt{snn\_input\_layer\_neurons\_size}.
    \item Segmentación en ventanas de longitud \texttt{T = 250} y \textit{padding} del conjunto de prueba.
\end{itemize}

\subsection{Configuración de hardware y software}
% TODO: Actualizar con la configuración real utilizada
\begin{itemize}
    \item \textbf{Hardware}:
    \begin{itemize}
        \item CPU: \textit{Intel(R) Core(TM) i7-14700HX}.
        \item GPU: \textit{NVIDIA GeForce RTX 4060 (8 GB VRAM)}.
        \item RAM: \textit{32 GB}.
        \item SO: \textit{Microsoft Windows 11 Pro (Build 26100)}.
    \end{itemize}
    \item \textbf{Software}:
    \begin{itemize}
        \item Python \textit{3.10.0}, PyTorch \textit{2.7.1+cpu}.
        \item NumPy \textit{2.1.3}, Pandas \textit{2.3.1}.
    \end{itemize}
    \item \textbf{Dispositivo}: CPU/GPU seleccionable vía \texttt{--device (cpu|gpu)}.

\end{itemize}

\subsection{Configuración de hardware y software}
\label{subsec:config_hw_sw}

En este Trabajo Fin de Máster se han utilizado dos entornos de trabajo principales:  El \textbf{Proyecto SNN} (con dos variantes internas: SNN original (A–B) y SNN híbrida (A–B–C)) y \textbf{Baselines TSFEDL}.

Ambos entornos no se unificaron deliberadamente debido a \textbf{incompatibilidades de versiones}, particularmente relacionadas con PyTorch, NumPy y Pandas. Intentar forzar una convergencia resultaba inviable no era posible encontrar una combinación de versiones que permitiera ejecutar simultáneamente ambos proyectos. A continuación se muestran las restricciones técnicas más relevantes:

\begin{itemize}
    \item \textbf{Restricción principal}: BindsNET 0.2.7 requiere una versión de PyTorch anterior a la 1.13, lo que lo hace incompatible con la rama 2.x utilizada por TSFEDL.
    \item \textbf{Efecto cadena}: librerías estrechamente ligadas, como \textit{Torchvision} y \textit{Lightning/torchmetrics}, demandan rangos de versiones distintos de \texttt{torch}, lo que agrava la incompatibilidad.
    \item \textbf{Cambios estructurales}: tanto NumPy 2.x como Pandas 2.x introducen modificaciones internas que obligarían a un proceso de refactorización antes de poder migrar el entorno SNN de manera estable.
\end{itemize}

\subsubsection*{Hardware (común a todos los experimentos)}
\begin{itemize}
    \item \textbf{CPU}: Intel(R) Core(TM) i7-14700HX
    \item \textbf{GPU}: NVIDIA GeForce RTX 4060 (8 GB VRAM)
    \item \textbf{RAM}: 32 GB
    \item \textbf{Sistema Operativo}: Microsoft Windows 11 Pro (Build 26100)
    \item \textbf{Dispositivo de cómputo}: Seleccionable mediante parámetro \texttt{--device \{cpu|gpu\}}
\end{itemize}

\subsubsection*{Comparación de versiones de software}
La Tabla~\ref{tab:comparativa_sw} resume las principales diferencias entre el entorno de TSFEDL y el del Proyecto SNN (aplicable a ambas variantes SNN: original y con capa convolucional).

\begin{table}[htbp]
\centering
\small
\begin{tabular}{lccc}
\hline\hline
\textbf{Librería} & \textbf{TSFEDL} & \textbf{Proyecto SNN} & \textbf{Observación} \\
\hline
Python        & 3.10.0     & 3.10.0     & Mismo \\
PyTorch       & 2.7.1+cpu  & 1.11.0     & Diferencia mayor (API / backend) \\
Torchvision   & --         & 0.12.0     & Sólo en SNN \\
NumPy         & 2.1.3      & 1.23.5     & Diferencia relevante (cambios internos) \\
Pandas        & 2.3.1      & 1.4.3      & Diferencia mayor (funciones depreciadas) \\
Matplotlib    & --         & 3.5.2      & Sólo en SNN (núcleo) \\
Scikit-learn  & --         & 1.1.1      & Sólo en SNN \\
BindsNET      & --         & 0.2.7      & Específico SNN (requiere PyTorch 1.x) \\
\hline\hline
\end{tabular}
\caption{Comparativa principal de versiones de software entre TSFEDL y Proyecto SNN.}
\label{tab:comparativa_sw}
\end{table}


\subsubsection*{Librerías adicionales del Proyecto SNN}
Además de las listadas en la tabla, el entorno SNN incluye:

\begin{description}
    \item[Aprendizaje Profundo / ML:] TensorFlow 2.19.0, Keras 3.10.0, PyTorch Lightning 2.5.2, Torchmetrics 1.7.4
    \item[Visualización:] Matplotlib 3.9.3 (instalada adicionalmente), Pillow 11.0.0, OpenCV 4.10.0.84
    \item[Procesamiento de datos:] SciPy 1.14.1, Scikit-image 0.24.0, NetworkX 3.4.2
    \item[Monitoreo / Logging:] Weights \& Biases (wandb) 0.21.0, TensorBoard 2.19.0, TensorBoardX 2.6.2.2
    \item[Utilidades:] Pytest 8.3.4, Rich 14.0.0, TQDM 4.67.1, Requests 2.32.3
    \item[Dominio de señales:] ObsPy 1.4.2 (sísmica), WFDB 4.3.0 (biomédica), SoundFile 0.13.1 (audio)
\end{description}


\subsubsection*{Estrategia de gestión de entornos}
Se emplearon entornos virtuales independientes utilizando \texttt{venv}. 

\subsection{Hiperparámetros y estrategia de búsqueda}
La SNN híbrida (A--B--C) se optimizó con Optuna (TPE) maximizando F1; en el código se minimiza \texttt{-best\_f1}. Espacios:
\begin{itemize}
    \item \textbf{Parámetros neuronales y STDP}:
    \begin{itemize}
        \item \texttt{nu1} \(\in [-0{,}5, 0{,}5]\), \texttt{nu2} \(\in [-0{,}5, 0{,}5]\).
        \item \texttt{threshold} \(\in [-65, -50]\), \texttt{decay} \(\in [80, 150]\).
    \end{itemize}
    \item \textbf{Capa convolucional (si activa)}:
    \begin{itemize}
        \item \texttt{kernel\_type} \(\in\) \{gaussian, laplacian, mexican\_hat, box\}.
        \item \texttt{kernel\_size} \(\in\) \{3,5,7,9\}.
        \item \texttt{sigma} \(\in [0{,}5, 3{,}0]\), \texttt{norm\_factor} \(\in [0{,}1, 1{,}0]\).
        \item \texttt{exc\_inh\_balance} \(\in [-0{,}3, 0{,}3]\).
        \item \texttt{conv\_processing\_type} \(\in\) \{direct, weighted\_sum, max\}.
    \end{itemize}
    \item \textbf{Parámetros fijos}: \texttt{T = 250}, \texttt{expansion = 100}, \texttt{a = 0{,}1}, \texttt{r = 0{,}05}, \texttt{snn\_process\_layer\_neurons\_size = 100}, \texttt{use\_conv\_layer = [True|False]}.
\end{itemize}
El número de \textit{trials} se controló con \texttt{--n\_trials} y se registró todo en W\&B.

\subsection{Resumen de configuraciones ensayadas}
% TODO: Añadir filas con los experimentos reales

\begin{table}[htbp]
\centering
\small
\begin{tabular}{llcccccc}
\hline\hline
\textbf{ExpID} & \textbf{Modelo} & \textbf{Dataset} & \textbf{Kernel} & \textbf{K} & \(\sigma\) & \textbf{Proc} & \texttt{T} \\
\hline
E0 & SNN (A--B) & IOPS & -- & -- & -- & -- & 250 \\
E1 & SNN (A--B--C) & IOPS & \textit{[TODO]} & \textit{[3/5/7/9]} & \textit{[TODO]} & \textit{[dir/ws/max]} & 250 \\
E2 & TSFEDL-\textit{[modelo]} & IOPS & N/A & N/A & N/A & N/A & -- \\
E3 & SNN (A--B) & CalIt2 & -- & -- & -- & -- & 250 \\
E4 & SNN (A--B--C) & CalIt2 & \textit{[TODO]} & \textit{[3/5/7/9]} & \textit{[TODO]} & \textit{[dir/ws/max]} & 250 \\
E5 & TSFEDL-\textit{[modelo]} & CalIt2 & N/A & N/A & N/A & N/A & -- \\
\hline\hline
\end{tabular}
\caption{Resumen de configuraciones ensayadas. K: tamaño de kernel; Proc: modo de procesamiento (direct/weighted\_sum/max).}
\label{tab:resumen-configuraciones}
\end{table}