\section{Análisis de resultados}

En esta sección se presentan y discuten los resultados, comparando la SNN original (A--B), la SNN híbrida (A--B--C) y modelos TSFEDL\footnote{\url{https://github.com/JGS9515/compare_to_TSFEDL}}. Se incluyen análisis por dataset, ablasiones y eficiencia.  Los tiempos de entrenamiento reportados deben interpretarse considerando que todos los modelos se ejecutaron en CPU (véase la Sección~\ref{subsec:entorno}).

\subsection{Resultados por dataset}

\subsubsection{IOPS}
% TODO: Añadir valores reales y comentarios
\begin{itemize}
    \item \textbf{Resumen}: la capa C del modelo híbrido \textit{[mejora/no mejora]} F1 respecto a la capa B y al modelo SNN base.
    \item \textbf{Tendencias}: \textit{[p.ej., kernels gaussian con weighted\_sum ofrecen mejor trade-off]}.
    \item \textbf{Errores}: \textit{[falsos positivos en transitorios / falsos negativos en anomalías suaves]}.
\end{itemize}

\begin{table}[htbp]
\centering
\small
\begin{tabular}{lcccccc}
\hline\hline
\textbf{Modelo} & \textbf{Prec} & \textbf{Rec} & \textbf{F1} & \textbf{MSE\_B} & \textbf{MSE\_C} & \textbf{Tiempo inf. (ms)} \\
\hline
SNN (A--B) & -- & -- & -- & -- & N/A & -- \\
SNN (A--B--C) & -- & -- & -- & -- & -- & -- \\
TSFEDL-\textit{[modelo]} & -- & -- & -- & N/A & N/A & -- \\
\hline\hline
\end{tabular}
\caption{Resultados en IOPS.}
\label{tab:resultados-iops}
\end{table}

\subsubsection{CalIt2}
% TODO: Añadir valores reales y comentarios
\begin{itemize}
    \item \textbf{Resumen}: \textit{[comentario sobre la mejora/empate]}.
    \item \textbf{Tendencias}: \textit{[p.ej., kernel\_size=5, sigma intermedio]}.
    \item \textbf{Errores}: \textit{[escenarios con eventos superpuestos]}.
\end{itemize}

\begin{table}[htbp]
\centering
\small
\begin{tabular}{lcccccc}
\hline\hline
\textbf{Modelo} & \textbf{Prec} & \textbf{Rec} & \textbf{F1} & \textbf{MSE\_B} & \textbf{MSE\_C} & \textbf{Tiempo inf. (ms)} \\
\hline
SNN (A--B) & -- & -- & -- & -- & N/A & -- \\
SNN (A--B--C) & -- & -- & -- & -- & -- & -- \\
TSFEDL-\textit{[modelo]} & -- & -- & -- & N/A & N/A & -- \\
\hline\hline
\end{tabular}
\caption{Resultados en CalIt2.}
\label{tab:resultados-calit2}
\end{table}

\subsection{Comparación con modelos TSFEDL}
% TODO: Enumerar modelos concretos de TSFEDL utilizados
Se compararon los siguientes modelos de la librería TSFEDL:
\begin{itemize}
    \item \textit{[CNN 1D / LSTM / GRU / TCN / Autoencoder ... (selección real empleada)]}.
\end{itemize}
Observaciones:
\begin{itemize}
    \item \textbf{Rendimiento}: \textit{[la SNN-C mejora F1 en X pp. / mantiene precisión con mejor recall]}.
    \item \textbf{Eficiencia}: \textit{[inferencias más rápidas/lentas que TSFEDL; coste computacional relativo]}.
\end{itemize}

\subsection{Importancia de hiperparámetros}
% TODO: Añadir figuras/tablas de Optuna si se dispone
\begin{itemize}
    \item \textbf{Capa convolucional}: impacto de \texttt{kernel\_type}, \texttt{kernel\_size}, \texttt{sigma}.
    \item \textbf{Procesamiento}: comparación entre \texttt{direct}, \texttt{weighted\_sum}, \texttt{max}.
    \item \textbf{Neuronales}: sensibilidad a \texttt{threshold} y \texttt{decay}.
\end{itemize}

\subsection{Eficiencia y escalabilidad}
% TODO: Rellenar con medidas de tiempo reales
\begin{itemize}
    \item \textbf{CPU vs GPU}: \textit{[ganancia observada en entrenamiento e inferencia]}.
    \item \textbf{Tiempo por ventana T=250}: \textit{[ms/ventana]}.
    \item \textbf{Memoria}: \textit{[pico de RAM/VRAM]}.
\end{itemize}

\subsection{Figuras recomendadas}
% TODO: Sustituir rutas por imágenes reales exportadas del pipeline
\begin{figure}[htbp]
    \centering
    \includegraphics[width=0.95\textwidth]{Imagenes/optuna_history_placeholder.png}
    \caption{Histórico de mejores valores por \textit{trial} (Optuna).}
    \label{fig:optuna-history}
\end{figure}

\begin{figure}[htbp]
    \centering
    \includegraphics[width=0.95\textwidth]{Imagenes/confusion_matrix_placeholder.png}
    \caption{Matriz de confusión en \textit{[dataset]} para \textit{[modelo]}.}
    \label{fig:confusion-matrix}
\end{figure}

\begin{figure}[htbp]
    \centering
    \includegraphics[width=0.95\textwidth]{Imagenes/pr_curve_placeholder.png}
    \caption{Curva Precisión-Recall comparativa.}
    \label{fig:pr-curve}
\end{figure}

\subsection{Amenazas a la validez}
\begin{itemize}
    \item \textbf{Particionado}: el uso de split 50/50 temporal puede no capturar todas las variaciones; considerar validación cruzada temporal.
    \item \textbf{Etiquetado}: expansión de etiquetas puede sesgar la estimación de \textit{recall}; justificar en función del objetivo de alertado temprano.
    \item \textbf{Recursos}: número limitado de \textit{trials} en Optuna por restricciones de tiempo/cómputo.
\end{itemize}

\subsection{Resumen}
% TODO: Redactar 3–5 bullets con hallazgos clave
\begin{itemize}
    \item \textit{[Hallazgo 1: mejora de F1 en dataset X con configuración Y]}.
    \item \textit{[Hallazgo 2: mejor trade-off precisión/recall con weighted\_sum]}.
    \item \textit{[Hallazgo 3: eficiencia en GPU/CPU y coste por ventana]}.
\end{itemize}